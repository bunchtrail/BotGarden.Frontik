\subsection{Реализация работы карты}

Основной компонент карты реализован в файле \texttt{MapPage.tsx} (см. приложение Б). Он представляет собой React-компонент, который управляет состоянием карты и координирует взаимодействие между различными подкомпонентами.

\begin{enumerate}
    \item Основные состояния компонента:
    \begin{enumerate}
        \item \texttt{mapImageURL} --- URL изображения карты.
        \item \texttt{currentMode} --- текущий режим работы карты.
        \item \texttt{areas} --- список областей.
        \item \texttt{markers} --- список маркеров.
    \end{enumerate}
\end{enumerate}

\subsubsection{Режимы работы карты}

Карта поддерживает несколько режимов работы, определённых в \texttt{mapControls.ts} (см. приложение Б):

\begin{enumerate}
    \item VIEW --- режим просмотра.
    \item ADD\_PLANT --- режим добавления растений.
    \item ADD\_AREA --- режим добавления областей.
    \item EDIT\_AREA --- режим редактирования областей.
    \item DELETE\_PLANTS\_IN\_AREA --- режим удаления растений в области.
\end{enumerate}

\subsubsection{Компоненты карты}

\subsubsection{MapView}
Компонент \texttt{MapView} (\texttt{MapView.tsx}, см. приложение Б) отвечает за:
\begin{enumerate}
    \item Отображение подложки карты.
    \item Управление маркерами растений.
    \item Управление областями.
    \item Обработку событий рисования и редактирования.
\end{enumerate}

\subsubsection{AreaPathModal}
Модальное окно \texttt{AreaPathModal} используется для ввода информации об области при её создании:
\begin{enumerate}
    \item Сохранение области.
    \item Отмена создания области.
\end{enumerate}

\subsubsection{Работа с сервером}

Взаимодействие с сервером осуществляется через сервисные функции, определённые в \texttt{mapService.ts} (см. приложение Б):
\begin{enumerate}
    \item \texttt{fetchMapImage()} --- получение изображения карты.
    \item \texttt{fetchAreas()} --- получение списка областей.
    \item \texttt{fetchMarkers()} --- получение списка маркеров.
    \item \texttt{addAreaToServer()} --- добавление новой области.
    \item \texttt{updateAreaOnServer()} --- обновление существующей области.
    \item \texttt{deleteAreaOnServer()} --- удаление области.
\end{enumerate}

\subsubsection{Поиск и фильтрация}

Реализована система поиска и фильтрации маркеров на карте:
\begin{enumerate}
    \item Ввод поискового запроса.
    \item Фильтрация маркеров по выбранным колонкам.
    \item Возврат списка маркеров, соответствующих критериям поиска.
\end{enumerate}

\subsubsection{Управление областями}

\subsubsection{Создание области}
При создании новой области используется следующий процесс:
\begin{enumerate}
    \item Пользователь активирует режим добавления области.
    \item Рисует полигон на карте.
    \item Вводит название области в модальном окне.
    \item Данные отправляются на сервер и добавляются в локальное состояние.
\end{enumerate}

\subsubsection{Редактирование области}
Редактирование существующих областей включает:
\begin{enumerate}
    \item Выбор области для редактирования.
    \item Обновление координат области.
    \item Сохранение изменений на сервере.
    \item Обновление локального состояния.
\end{enumerate}

\subsubsection{Загрузка изображения карты}

Реализована возможность загрузки нового изображения карты:
\begin{enumerate}
    \item Выбор файла изображения.
    \item Загрузка файла на сервер.
    \item Обновление URL изображения в состоянии приложения.
\end{enumerate}
